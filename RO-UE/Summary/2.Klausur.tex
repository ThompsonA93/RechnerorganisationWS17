% Preamble
\documentclass[11pt]{article}

    % Packages
    \usepackage{a4wide}
    \usepackage[utf8]{inputenc}
    \usepackage[T1]{fontenc}
    \usepackage[naustrian]{babel}
    
    \usepackage{verbatim}
    \usepackage{enumerate}
    \usepackage{scrextend}
    \usepackage{graphicx}
    
    \title{Rechnerorganisation - 2. Klausurvorbereitung}
    \author{Auer Thomas}
    \date{\today}

\begin{document}
\maketitle
\tableofcontents
\graphicspath{{graphics/}}

\section{Übungsblatt 5}
    \subsection{Multi-Cycle Datenpfad: lw-Instruktion}
    \textbf{Erläutern Sie die Ausführung der lw-Instruktion (load word) für den Multi-Cycle-Datenpfad.
    Welche Vorteile bietet der Multi-Cycle-Datenpfad gegenüber dem Single-Cycle-Datenpfad?
    Diskutieren Sie dies anhand dieses Befehls.}

    \subsection{Grundlagen Pipelining}
    \textbf{Gegeben seien vier unterschiedliche Prozessoren, die sich in der Anzahl der Pipelinestufen und
    der Taktrate unterscheiden:\\}
    \begin{center}
        \begin{tabular}{l|l|l}
        Prozessor & Pipelinestufen & Taktrate \\\hline
        A & 1 & 100 MHz \\
        B & 4 & 800 MHz \\
        C & 12 & 1,5 GHz \\
        D & 20 & 3,2 GHz \\
        \end{tabular}
    \end{center}
    \textbf{(a) Bestimmen Sie für jeden Prozessor die Latenz der einzelnen Instruktionen.}
    
    \textbf{Wie lange dauert die Ausführung von 400.000 voneinander unabhängigen Instruktionen
    auf jedem der angeführten Prozessoren? Bestimmen Sie die Performance und den
    Speedup verglichen mit Prozessor A ohne Pipelining. (Sie können annehmen, dass es
    keine Stalls gibt.)}

    \subsection{Pipelining: Graphische Darstellung}

    \textbf{Beantworten Sie folgende Fragen anhand der Beispiel-Pipeline-Architektur der VO (Kapitel 3.2).\\\\
    (a) Bestimmen Sie die Anzahl der Pipelinestufen, die Taktdauer und die Taktfrequenz der
    Beispiel-Pipeline unter Annahme der Angaben auf VO-Folie 3-44 (Ausführungszeiten der
    Funktionseinheiten). Wie lange dauert die Ausführung eines einzigen Befehls auf der
    Beispiel-Pipeline?\\\\
    (b) Angenommen es treten keine Leertakte (stalls) auf, welchen Speedup erreicht die
    Beispiel-Pipeline aus a) gegenüber einem Single-Cycle Datenpfad, der aus den gleichen
    Funktionsregistern besteht?\\
    Seite 1 von 2\\\\
    (c) Auf der Pipeline wird folgende Befehlssequenz ausgeführt:\\}
    \begin{verbatim}
        and     $10, $2, $3
        sw      $11, 4($3)
    \end{verbatim}

    \textbf{    Stellen Sie die Ausführung der oben angeführten Befehlssequenz durch die Beispiel-Pipeline
    wie auf VO-Folie 3-45 grafisch dar (untere Abbildung). Achten Sie insbesondere auf die
    zeitliche Anordnung der Zugriffe auf die Registereinheit! Wie lange dauert die Ausführung der
    Befehlssequenz?}


    \subsection{Pipelining: Daten- und Kontrollabhängigkeiten}
    \textbf{Gegeben sei folgendes Code-Fragment:}
    \begin{verbatim}
        
        
    \end{verbatim}
    


    \section{Übungsblatt 6}


\section{Übungsblatt 7}


\section{Übungsblatt 8}


\section{Übungsblatt 9}


\section{Übungsblatt 10}

    
\end{document}