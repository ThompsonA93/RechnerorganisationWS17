% Preamble
\documentclass[11pt]{scrartcl}

    % Packages
    \usepackage{a4wide}
    \usepackage[utf8]{inputenc}
    \usepackage[T1]{fontenc}
    \usepackage[naustrian]{babel}
    \usepackage{lmodern}
    \usepackage{microtype}
    
    \usepackage{verbatim}
    \usepackage{enumerate}
    \usepackage{scrextend}
    \usepackage{graphicx}
    \usepackage{parskip}
   	\usepackage[dvipsnames]{xcolor}
    \usepackage{tikz}
   
    \usepackage{amsmath}
    \usepackage{amsfonts}
    \usepackage{amssymb}
        
    \title{Rechnerorganisation - 2.\ Klausurvorbereitung}
    \author{Auer Thomas, Stefan Haan}
    \date{\today}

\begin{document}
\maketitle
\tableofcontents
\graphicspath{{graphics/}}

\setcounter{section}{4}
\pagebreak

\section{Übungsblatt 5}
    \subsection{Multi-Cycle Datenpfad: lw-Instruktion}
    \textbf{Erläutern Sie die Ausführung der lw-Instruktion (load word) für den Multi-Cycle-Datenpfad.
    Welche Vorteile bietet der Multi-Cycle-Datenpfad gegenüber dem Single-Cycle-Datenpfad?
    Diskutieren Sie dies anhand dieses Befehls.}

  %  \includegraphics[width=\textwidth]{LW_MCDp}

    \subsection{Grundlagen Pipelining}
    \textbf{Gegeben seien vier unterschiedliche Prozessoren, die sich in der Anzahl der Pipelinestufen und
    der Taktrate unterscheiden:\\}
    \begin{center}
        \begin{tabular}{l|l|l}
        Prozessor & Pipelinestufen & Taktrate \\\hline
        A & 1 & 100 MHz \\
        B & 4 & 800 MHz \\
        C & 12 & 1,5 GHz \\
        D & 20 & 3,2 GHz \\
        \end{tabular}
    \end{center}
    \textbf{(a) Bestimmen Sie für jeden Prozessor die Latenz der einzelnen Instruktionen.}
    
    \textbf{Wie lange dauert die Ausführung von 400.000 voneinander unabhängigen Instruktionen
    auf jedem der angeführten Prozessoren? Bestimmen Sie die Performance und den
    Speedup verglichen mit Prozessor A ohne Pipelining. (Sie können annehmen, dass es
    keine Stalls gibt.)}

    \subsection{Pipelining: Graphische Darstellung}

    \textbf{Beantworten Sie folgende Fragen anhand der Beispiel-Pipeline-Architektur der VO (Kapitel 3.2).\\\\
    (a) Bestimmen Sie die Anzahl der Pipelinestufen, die Taktdauer und die Taktfrequenz der
    Beispiel-Pipeline unter Annahme der Angaben auf VO-Folie 3-44 (Ausführungszeiten der
    Funktionseinheiten). Wie lange dauert die Ausführung eines einzigen Befehls auf der
    Beispiel-Pipeline?\\\\
    (b) Angenommen es treten keine Leertakte (stalls) auf, welchen Speedup erreicht die
    Beispiel-Pipeline aus a) gegenüber einem Single-Cycle Datenpfad, der aus den gleichen
    Funktionsregistern besteht?\\
    Seite 1 von 2\\\\
    (c) Auf der Pipeline wird folgende Befehlssequenz ausgeführt:\\}
    \begin{verbatim}
        and     $10, $2, $3
        sw      $11, 4($3)
    \end{verbatim}

    \textbf{    Stellen Sie die Ausführung der oben angeführten Befehlssequenz durch die Beispiel-Pipeline
    wie auf VO-Folie 3-45 grafisch dar (untere Abbildung). Achten Sie insbesondere auf die
    zeitliche Anordnung der Zugriffe auf die Registereinheit! Wie lange dauert die Ausführung der
    Befehlssequenz?}


    \subsection{Pipelining: Daten- und Kontrollabhängigkeiten}
    \textbf{Gegeben sei folgendes Code-Fragment:}

    \begin{center}
        
%    \includegraphics[width=0.5\textwidth]{5_4}
    
    \end{center}


\pagebreak
\section{Übungsblatt 6}
\setcounter{subsection}{4}
\subsection{Loop Unrolling}
Folgendes Codefragment wird auf einen Prozessor mit "`Delayed Branching"' (1 Takt Branch Delay) ausgeführt. Die Latenzen zwischen abhängigen Befehlen sind in Tabelle \ref{tbl:stalls1} aufgelistet.

{
	\ttfamily
	\begin{tabular}{l llll}
		loop: & l.d   & \$f4  & 0(\$t0) &      \\
		      & sub.d & \$f6  & \$f4    & \$f0 \\
		      & l.d   & \$f8  & 0(\$t1) &      \\
		      & mul.d & \$f10 & \$f6    & \$f8 \\
		      & add.d & \$f12 & \$f10   & \$f2 \\
		      & s.d   & \$f12 & 0(\$t2) &      \\
		      & addi  & \$t2  & \$t2    & -8   \\
		      & addi  & \$t1  & \$t1    & -8   \\
		      & addi  & \$t0  & \$t0    & -8   \\
		      & bne   & \$t0  & \$t4    & loop \\
		      & nop   &       &         &
	\end{tabular}
}

\begin{table}[h!]
	\centering
	\begin{tabular}{lll}
		\hline
		Erzeugender Befehl & Benutzender Befehl & Zwischentakte \\ \hline
		FP ALU operation   & FP ALU operation   & 3             \\
		FP ALU operation   & Store FP double    & 2             \\
		Load FP double     & FP ALU operation   & 1             \\
		Load FP double     & Store FP double    & 0             \\
		Load integer       & Integer operation  & 1             \\
		Load integer       & Branch             & 2             \\
		Integer operation  & Integer operation  & 0             \\
		Integer operation  & Branch             & 1             \\ \hline
	\end{tabular}
	\caption{Latenzen zwischen abhängigen Befehlen}
	\label{tbl:stalls1}
\end{table}

\begin{enumerate}[(a)]
	\item \textbf{Identifizieren Sie alle Daten- und Kontrollabhängigkeiten, die Stalls verursachen. Wie viele Takte werden für ein Ergebniselement\footnote{durch s.d gespeicherter Wert} benötigt?}

	{
		\ttfamily
		\begin{tabular}{l llll}
			loop:                       & l.d   & \color{blue}\$f4        & 0(\$t0)               &                  \\
			\color{blue} +1 Stall       & sub.d & \color{brown}\$f6       & \color{blue}\$f4      & \$f0             \\
			                            & l.d   & \color{cyan}\$f8        & 0(\$t1)               &                  \\
			\color{brown}+2 Stalls      & mul.d & \color{Mulberry}\$f10   & \color{brown}\$f6     & \color{cyan}\$f8 \\
			\color{Mulberry}+3 Stalls   & add.d & \color{Orange}\$f12     & \color{Mulberry}\$f10 & \$f2             \\
			\color{Orange}+2 Stalls     & s.d   & \color{Orange}\$f12     & 0(\$t2)               &                  \\
			                            & addi  & \$t2                    & \$t2                  & -8               \\
			                            & addi  & \$t1                    & \$t1                  & -8               \\
			                            & addi  & \color{Bittersweet}\$t0 & \$t0                  & -8               \\
			\color{Bittersweet}+1 Stall & bne   & \color{Bittersweet}\$t0 & \$t4                  & loop             \\
			                            & nop   &                         &                       &
		\end{tabular}
	}
	\begin{tabular}{llll}
		\hline
		                & Stalls & Instruktionen & $ \Sigma $ \\ \hline
		Ergebniselement & 9      & 11            & 20         \\ \hline
	\end{tabular}

	\pagebreak	
	\item \textbf{Optimieren Sie den Code durch Umordnen von Befehlen so, dass er auf dem gegebenen Prozessor möglichst schnell ausgeführt wird. Wie viele Takte werden für die Verarbeitung eines Ergebniselements benötigt?}
	
{
	\ttfamily
	\begin{tabular}{l llll}
		loop:                        & l.d   & \$f4                     & 0(\$t0)               &                \\
		                             & l.d   & \$f8                     & 0(\$t1)               &                \\
		                             & sub.d & \color{Maroon}\$f6       & \$f4                  & \$f0           \\
		                             & addi  & \$t2                     & \$t2                  & \color{red} -8 \\
		                             & addi  & \$t1                     & \$t1                  & -8             \\
		\color{Maroon}+1 Stall       & mul.d & \color{NavyBlue}\$f10    & \color{Maroon}\$f6    & \$f8           \\
		                             & addi  & \$t0                     & \$t0                  & -8             \\
		\color{NavyBlue}+2 Stalls    & add.d & \color{ForestGreen}\$f12 & \color{NavyBlue}\$f10 & \$f2           \\
		\color{ForestGreen}+1 Stalls & bne   & \$t0                     & \$t4                  & loop           \\
		                             & s.d   & \color{ForestGreen}\$f12 & \color{red}8(\$t2)    &
	\end{tabular}
}
	\begin{tabular}{llll}
		\hline
		                & Stalls & Instruktionen & $ \Sigma $ \\ \hline
		Ergebniselement & 4      & 10            & 14         \\ \hline
	\end{tabular}


\item \textbf{Rollen Sie die Schleife zweimal ab und ordnen Sie den Code so um, dass er auf dem gegebenen Prozessor möglichst schnell ausgeführt wird. Wie viele Takte werden nun pro	Ergebniselement benötigt?}

{
	\ttfamily
	\begin{tabular}{l llll}
		loop: & l.d   & \$f4  & 0(\$t0)             &                \\
		      & l.d   & \$f14 & -8(\$t0)            &                \\
		      & sub.d & \$f6  & \$f4                & \$f0           \\
		      & sub.d & \$f16 & \$f14               & \$f0           \\
		      & l.d   & \$f8  & 0(\$t1)             &                \\
		      & l.d   & \$f18 & -8(\$t1)            &                \\
		      & mul.d & \$f10 & \$f6                & \$f8           \\
		      & mul.d & \$f20 & \$f16               & \$f8           \\
		      & addi  & \$t2  & \$t2                & \color{red}-16 \\
		      & addi  & \$t1  & \$t1                & \color{red}-16 \\
		      & add.d & \$f12 & \$f10               & \$f2           \\
		      & add.d & \$f22 & \$f20               & \$f22          \\
		      & addi  & \$t0  & \$t0                & -16            \\
		      & s.d   & \$f12 & \color{red}16(\$t2) &                \\
		      & bne   & \$t0  & \$t4                & loop           \\
		      & s.d   & \$f22 & \color{red}8(\$t2)  &
	\end{tabular}
}
	\begin{tabular}{llll}
		\hline
		           & Stalls & Instruktionen & $ \Sigma $ \\ \hline
		2 Elemente & 0      & 16            & 16         \\
		1 Element  & 0      & 8             & 8          \\ \hline
	\end{tabular}

\end{enumerate}

\section{Übungsblatt 7}


\section{Übungsblatt 8}
\setcounter{subsection}{3}
\subsection{Speicherhierarchien 1}
Tabelle \ref{tbl:hirachy1} gibt die \textbf{Verzögerung} im Falle eines \textbf{erfolgreichen} Zugriffs auf die jeweilige Speicherebene an.

Der durchschnittliche CPI-Wert \underline{ohne} Berücksichtigung der Speicherzugriffe betrage 1,3 (idealer CPI). \textbf{20\%} aller Instruktionen greifen auf den Speicher zu.

\begin{table}[h!]
	\centering
	\begin{tabular}{lll}
		\hline
		Speicherebene & Hitrate & Verzögerung/Takte \\\hline
		L1 & 85\% & 4 \\
		L2 & 75\% & 12 \\
		RAM & 100\% & 236 \\\hline
	\end{tabular}
	\caption{Speicherebenen und Verzögerung in Takten}
	\label{tbl:hirachy1}
\end{table}

\begin{enumerate}[a)]
	\item \textbf{Durchschnittlicher CPI-Wert unter Berücksichtigung der Speicherzugriffe}
	\begin{figure}[h!]
		\centering
		\begin{tikzpicture}
		\node{Instruktionen}[->]{
			child{node{Speicherzugriffe} 
				child{node{L1 Miss}
					child{node{L2 Miss}
						child{node[align=center]{RAM\\\textbf{236 CPI}}}
						edge from parent node[left=3mm]{0.25}
					}
					child[missing]
					child{node[align=center]{L2 Hit\\\textbf{12 CPI}}
						edge from parent node[right=3mm]{0.75}
					}
					edge from parent node[left=3mm]{0.15}
				}
				child[missing]
				child{node[align=center]{L1 Hit\\\textbf{4 CPI}}
					edge from parent node[right=3mm]{0.85}
				}
				edge from parent node[left=3mm]{0.20}
			}
			child[missing]
			child{node[align=center]{Andere\\\textbf{0 CPI}}
				edge from parent node[right=3mm]{0.80}
			}
		};
		\end{tikzpicture}
		\caption{Wahrscheinlichkeiten der Speicherzugriffe mit ihren \textbf{Verzögerungen}}
		\label{fig:prob1}
	\end{figure}

	Berechne die durchschnittliche Verzögerung aus Abbildung \ref{fig:prob1} durch gewichtete Aufsummierung der Knoten.
	\begin{equation}
		\overline{\text{Verzögerung}} = 0.8 \cdot 0 + 0.2 \cdot (0.85 \cdot 4 + 0.15 \cdot(0.75 \cdot 12 + 0.25 \cdot 236)) = 2.72
		\label{eq:avgdelay1}
	\end{equation}
	
	
	Addiere die durchschnittliche Verzögerung zum idealen CPI-Wert um den durchschnittlichen CPI-Wert unter Berücksichtigung der Speicherzugriffe zu erhalten.
	\begin{equation}
		\overline{\text{CPI}_\text{gesamt}} = \overline{\text{Verzögerung}} + 1.3 = 4.02
	\end{equation}
	
	
	\item \textbf{Prozent der Ausführungszeit, die der Prozessor auf Speicherzugriffe warten muss}
	
	Die durch Speicherzugriffe verursachte Verzögerung ist $ \overline{\text{Verzögerung}} = 2.72 $. Berechne das Verhältnis zum durchschnittlichen CPI-Wert.
	\begin{equation}
		\frac{\overline{\text{Verzögerung}}}{\overline{\text{CPI}_\text{gesamt}}} = \frac{2.72}{4.02} \approx 0.677
	\end{equation}
	
	
	\item \textbf{Hitrate des L1-Caches, um einen durchschnittlichen CPI-Wert von 3 zu erhalten}
	
	Die maximale Verzögerung, die nun durch Speicherzugriffe auftreten darf, ist $ \overline{\text{Verzögerung}} = 3 - 1.3 = 1.7 $.
	Man schreibe die Gleichung \ref{eq:avgdelay1} um und löse nach der Hitrate des L1 Caches $ p_\text{L1} $.
	\begin{align}
	\overline{\text{Verzögerung}} = 0.8 \cdot 0 + 0.2 \cdot (p_\text{L1} \cdot 4 + (1- p_\text{L1}) \cdot(0.75 \cdot 12 + 0.25 \cdot 236)) &= 1.7 \\
	 p_\text{L1} \cdot 4 + (1- p_\text{L1}) \cdot 68 &= \frac{1.7}{0.2} \\
	  4 \cdot p_\text{L1} - 68 \cdot p_\text{L1} &= 8.5 - 68 \\
	  p_\text{L1} &= \frac{-59.5}{-64} \\
	  &\approx 0.93
	\end{align}
\end{enumerate}


\section{Übungsblatt 9}


\section{Übungsblatt 10}

    
\end{document}