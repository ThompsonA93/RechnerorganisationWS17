\pagebreak
\section{Übungsblatt 6}
\setcounter{subsection}{4}
\subsection{Loop Unrolling}
Folgendes Codefragment wird auf einen Prozessor mit "`Delayed Branching"' (1 Takt Branch Delay) ausgeführt. Die Latenzen zwischen abhängigen Befehlen sind in Tabelle \ref{tbl:stalls1} aufgelistet.

{
	\ttfamily
	\begin{tabular}{l llll}
		loop: & l.d   & \$f4  & 0(\$t0) &      \\
		      & sub.d & \$f6  & \$f4    & \$f0 \\
		      & l.d   & \$f8  & 0(\$t1) &      \\
		      & mul.d & \$f10 & \$f6    & \$f8 \\
		      & add.d & \$f12 & \$f10   & \$f2 \\
		      & s.d   & \$f12 & 0(\$t2) &      \\
		      & addi  & \$t2  & \$t2    & -8   \\
		      & addi  & \$t1  & \$t1    & -8   \\
		      & addi  & \$t0  & \$t0    & -8   \\
		      & bne   & \$t0  & \$t4    & loop \\
		      & nop   &       &         &
	\end{tabular}
}

\begin{table}[h!]
	\centering
	\begin{tabular}{lll}
		\hline
		Erzeugender Befehl & Benutzender Befehl & Zwischentakte \\ \hline
		FP ALU operation   & FP ALU operation   & 3             \\
		FP ALU operation   & Store FP double    & 2             \\
		Load FP double     & FP ALU operation   & 1             \\
		Load FP double     & Store FP double    & 0             \\
		Load integer       & Integer operation  & 1             \\
		Load integer       & Branch             & 2             \\
		Integer operation  & Integer operation  & 0             \\
		Integer operation  & Branch             & 1             \\ \hline
	\end{tabular}
	\caption{Latenzen zwischen abhängigen Befehlen}
	\label{tbl:stalls1}
\end{table}

\begin{enumerate}[(a)]
	\item \textbf{Identifizieren Sie alle Daten- und Kontrollabhängigkeiten, die Stalls verursachen. Wie viele Takte werden für ein Ergebniselement\footnote{durch s.d gespeicherter Wert} benötigt?}

	{
		\ttfamily
		\begin{tabular}{l llll}
			loop:                       & l.d   & \color{blue}\$f4        & 0(\$t0)               &                  \\
			\color{blue} +1 Stall       & sub.d & \color{brown}\$f6       & \color{blue}\$f4      & \$f0             \\
			                            & l.d   & \color{cyan}\$f8        & 0(\$t1)               &                  \\
			\color{brown}+2 Stalls      & mul.d & \color{Mulberry}\$f10   & \color{brown}\$f6     & \color{cyan}\$f8 \\
			\color{Mulberry}+3 Stalls   & add.d & \color{Orange}\$f12     & \color{Mulberry}\$f10 & \$f2             \\
			\color{Orange}+2 Stalls     & s.d   & \color{Orange}\$f12     & 0(\$t2)               &                  \\
			                            & addi  & \$t2                    & \$t2                  & -8               \\
			                            & addi  & \$t1                    & \$t1                  & -8               \\
			                            & addi  & \color{Bittersweet}\$t0 & \$t0                  & -8               \\
			\color{Bittersweet}+1 Stall & bne   & \color{Bittersweet}\$t0 & \$t4                  & loop             \\
			                            & nop   &                         &                       &
		\end{tabular}
	}
	\begin{tabular}{llll}
		\hline
		                & Stalls & Instruktionen & $ \Sigma $ \\ \hline
		Ergebniselement & 9      & 11            & 20         \\ \hline
	\end{tabular}

	\pagebreak	
	\item \textbf{Optimieren Sie den Code durch Umordnen von Befehlen so, dass er auf dem gegebenen Prozessor möglichst schnell ausgeführt wird. Wie viele Takte werden für die Verarbeitung eines Ergebniselements benötigt?}
	
{
	\ttfamily
	\begin{tabular}{l llll}
		loop:                        & l.d   & \$f4                     & 0(\$t0)               &                \\
		                             & l.d   & \$f8                     & 0(\$t1)               &                \\
		                             & sub.d & \color{Maroon}\$f6       & \$f4                  & \$f0           \\
		                             & addi  & \$t2                     & \$t2                  & \color{red} -8 \\
		                             & addi  & \$t1                     & \$t1                  & -8             \\
		\color{Maroon}+1 Stall       & mul.d & \color{NavyBlue}\$f10    & \color{Maroon}\$f6    & \$f8           \\
		                             & addi  & \$t0                     & \$t0                  & -8             \\
		\color{NavyBlue}+2 Stalls    & add.d & \color{ForestGreen}\$f12 & \color{NavyBlue}\$f10 & \$f2           \\
		\color{ForestGreen}+1 Stalls & bne   & \$t0                     & \$t4                  & loop           \\
		                             & s.d   & \color{ForestGreen}\$f12 & \color{red}8(\$t2)    &
	\end{tabular}
}
	\begin{tabular}{llll}
		\hline
		                & Stalls & Instruktionen & $ \Sigma $ \\ \hline
		Ergebniselement & 4      & 10            & 14         \\ \hline
	\end{tabular}


\item \textbf{Rollen Sie die Schleife zweimal ab und ordnen Sie den Code so um, dass er auf dem gegebenen Prozessor möglichst schnell ausgeführt wird. Wie viele Takte werden nun pro	Ergebniselement benötigt?}

{
	\ttfamily
	\begin{tabular}{l llll}
		loop: & l.d   & \$f4  & 0(\$t0)             &                \\
		      & l.d   & \$f14 & -8(\$t0)            &                \\
		      & sub.d & \$f6  & \$f4                & \$f0           \\
		      & sub.d & \$f16 & \$f14               & \$f0           \\
		      & l.d   & \$f8  & 0(\$t1)             &                \\
		      & l.d   & \$f18 & -8(\$t1)            &                \\
		      & mul.d & \$f10 & \$f6                & \$f8           \\
		      & mul.d & \$f20 & \$f16               & \$f8           \\
		      & addi  & \$t2  & \$t2                & \color{red}-16 \\
		      & addi  & \$t1  & \$t1                & \color{red}-16 \\
		      & add.d & \$f12 & \$f10               & \$f2           \\
		      & add.d & \$f22 & \$f20               & \$f22          \\
		      & addi  & \$t0  & \$t0                & -16            \\
		      & s.d   & \$f12 & \color{red}16(\$t2) &                \\
		      & bne   & \$t0  & \$t4                & loop           \\
		      & s.d   & \$f22 & \color{red}8(\$t2)  &
	\end{tabular}
}
	\begin{tabular}{llll}
		\hline
		           & Stalls & Instruktionen & $ \Sigma $ \\ \hline
		2 Elemente & 0      & 16            & 16         \\
		1 Element  & 0      & 8             & 8          \\ \hline
	\end{tabular}

\end{enumerate}